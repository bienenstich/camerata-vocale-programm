\documentclass[11pt, ngermanm, titlepage]{article}
\usepackage[ngerman]{babel}
\usepackage{fancyhdr}
\usepackage{multicol}
\usepackage{a4wide}
\usepackage{tabu, tabularx}

\setlength{\columnsep}{1cm}
\setlength{\columnwidth}{10cm}
\setlength{\extrarowheight}{15pt}

\begin{document}
	\pagenumbering{gobble}
	\begin{titlepage}
		\pagestyle{empty}
		\title{In dulci jubilo}
		\date{08.12.2018}
		\author{Camerata Vocale München, Ltg.: Clayton Bowman}
		\maketitle	
	\end{titlepage}
	\pagebreak
	\quad
	\pagebreak
	\section*{Programm}
	\begin{tabularx} \textwidth {lX}
		Francis Poulenc (1899 - 1963) & Quatre motets pour le temps de No\"el: \newline "`O Magnum Mysterium"'
		\\
		Thomas Weelkes (1576 - 1623) & "`Hosanna to the Son of David"' 
		\\
		Francis Poulenc (1899 - 1963) & Quatre motets pour le temps de No\"el: \newline "`Videntes Stellam"' 
		\\
		Andreas Hammerschmidt (1611 - 1675) & "`Machet die Tore weit"' 
		\\
		Francis Poulenc (1899 - 1963) & Quatre motets pour le temps de No\"el: \newline "`Quem Vidistis Pastores Dicite"' 
		\\
		Michael Praetorius (1571 - 1621) & "`In dulci jubilo"'
		\\
		Francis Poulenc (1899 - 1963) & Quatre motets pour le temps de No\"el: \newline "`Hodie Christus Natus Est"'
		\\
		Randall Thompson (1899 - 1984) & "`Alleluia"'
	\end{tabularx}
	\pagebreak
	\begin{multicols}{2}
	\section*{Komponist}
	\paragraph{Titel (1738)\newline}
	Nach einem dreijährigen Aufenthalt in New York kehrte Benjamin Britten im Jahre 1942 in seine Heimat England zurück. Während der Heimreise auf einem schwedischen Frachtschiff komponierte er neben anderen Werken seine "`Hymn to St. Cecilia"'. Zu Ehren der Patronin der Kirchenmusik, der Heiligen Cäcilia von Rom, waren bereits seit dem 17. und 18. Jahrhundert zahlreiche Werke entstanden. Britten führte diese Tradition fort und vertonte einen Text des Schriftstellers W. H. Auden, der sein Wirken zeit seines Lebens nicht nur auf persönlicher, sondern auch auf künstlerischer Ebene stark beeinflusste. 

	Ganz in der Tradition vorhergegangener Cäcilien-Loblieder schließt der erste anmutig schwebende Teil der dreigeteilten Hymne mit einer feierlichen Anrufung der Seligen: "`Blessed Cecilia, appear in visions to all musicians."' Diese Verse kehren nach jeder Sequenz musikalisch variiert, doch refrainartig wieder. Im zweiten Abschnitt, einem leichten, schnelleren Scherzo, spielen sich Soprane und Tenöre die Worte zu, während Alti und Bässe denselben Text als cantus firmus darbieten. Der Höhepunkt erfolgt im instrumental charakterisierten Schlussteil, in dem Soli aus jeder Stimme den Klang einzelner Instrumente beschreiben. Die letzte Invokation der Heiligen schließt mit einer erhabenen und friedvollen Kadenz in E-Dur. 
	
	\section*{Komponist}
	\paragraph{Titel (1738)\newline}
	Nach einem dreijährigen Aufenthalt in New York kehrte Benjamin Britten im Jahre 1942 in seine Heimat England zurück. Während der Heimreise auf einem schwedischen Frachtschiff komponierte er neben anderen Werken seine "`Hymn to St. Cecilia"'. Zu Ehren der Patronin der Kirchenmusik, der Heiligen Cäcilia von Rom, waren bereits seit dem 17. und 18. Jahrhundert zahlreiche Werke entstanden. Britten führte diese Tradition fort und vertonte einen Text des Schriftstellers W. H. Auden, der sein Wirken zeit seines Lebens nicht nur auf persönlicher, sondern auch auf künstlerischer Ebene stark beeinflusste. 
	
	Ganz in der Tradition vorhergegangener Cäcilien-Loblieder schließt der erste anmutig schwebende Teil der dreigeteilten Hymne mit einer feierlichen Anrufung der Seligen: "`Blessed Cecilia, appear in visions to all musicians."' Diese Verse kehren nach jeder Sequenz musikalisch variiert, doch refrainartig wieder. Im zweiten Abschnitt, einem leichten, schnelleren Scherzo, spielen sich Soprane und Tenöre die Worte zu, während Alti und Bässe denselben Text als cantus firmus darbieten. Der Höhepunkt erfolgt im instrumental charakterisierten Schlussteil, in dem Soli aus jeder Stimme den Klang einzelner Instrumente beschreiben. Die letzte Invokation der Heiligen schließt mit einer erhabenen und friedvollen Kadenz in E-Dur. 
	
	\section*{Komponist}
	\paragraph{Titel (1738)\newline}
	Nach einem dreijährigen Aufenthalt in New York kehrte Benjamin Britten im Jahre 1942 in seine Heimat England zurück. Während der Heimreise auf einem schwedischen Frachtschiff komponierte er neben anderen Werken seine "`Hymn to St. Cecilia"'. Zu Ehren der Patronin der Kirchenmusik, der Heiligen Cäcilia von Rom, waren bereits seit dem 17. und 18. Jahrhundert zahlreiche Werke entstanden. Britten führte diese Tradition fort und vertonte einen Text des Schriftstellers W. H. Auden, der sein Wirken zeit seines Lebens nicht nur auf persönlicher, sondern auch auf künstlerischer Ebene stark beeinflusste. 
	
	Ganz in der Tradition vorhergegangener Cäcilien-Loblieder schließt der erste anmutig schwebende Teil der dreigeteilten Hymne mit einer feierlichen Anrufung der Seligen: "`Blessed Cecilia, appear in visions to all musicians."' Diese Verse kehren nach jeder Sequenz musikalisch variiert, doch refrainartig wieder. Im zweiten Abschnitt, einem leichten, schnelleren Scherzo, spielen sich Soprane und Tenöre die Worte zu, während Alti und Bässe denselben Text als cantus firmus darbieten. Der Höhepunkt erfolgt im instrumental charakterisierten Schlussteil, in dem Soli aus jeder Stimme den Klang einzelner Instrumente beschreiben. Die letzte Invokation der Heiligen schließt mit einer erhabenen und friedvollen Kadenz in E-Dur. 
	\end{multicols}
\end{document}