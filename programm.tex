\documentclass[11pt, ngermanm, titlepage]{article}
\usepackage[ngerman]{babel}
\usepackage{fancyhdr}
\usepackage{multicol}
\usepackage{a4wide}
\usepackage{tabu, tabularx}

\renewcommand{\rmdefault}{phv}


\setlength{\columnsep}{1cm}
\setlength{\columnwidth}{10cm}
\setlength{\extrarowheight}{15pt}

\begin{document}
	\pagenumbering{gobble}
	\begin{titlepage}
		\pagestyle{empty}
		\title{In dulci jubilo}
		\date{08.12.2018}
		\author{Camerata Vocale München, Ltg.: Clayton Bowman}
		\maketitle	
	\end{titlepage}
	\pagebreak
	\quad
	\pagebreak
	\section*{Programm}
	\begin{tabularx} \textwidth {lX}
		Francis Poulenc (1899 - 1963) & Quatre motets pour le temps de No\"el: \newline "`O Magnum Mysterium"'
		\\
		Thomas Weelkes (1576 - 1623) & "`Hosanna to the Son of David"' 
		\\
		Francis Poulenc (1899 - 1963) & Quatre motets pour le temps de No\"el: \newline "`Videntes Stellam"' 
		\\
		Andreas Hammerschmidt (1611 - 1675) & "`Machet die Tore weit"' 
		\\
		Francis Poulenc (1899 - 1963) & Quatre motets pour le temps de No\"el: \newline "`Quem Vidistis Pastores Dicite"' 
		\\
		Michael Praetorius (1571 - 1621) & "`In dulci jubilo"'
		\\
		Francis Poulenc (1899 - 1963) & Quatre motets pour le temps de No\"el: \newline "`Hodie Christus Natus Est"'
		\\
		Randall Thompson (1899 - 1984) & "`Alleluia"'
	\end{tabularx}
	\pagebreak
	\begin{multicols}{2}
	\section*{Andreas Hammerschmidt}
	\paragraph{(geboren 1611 oder 1612 in Brüx, Böhmen; gestorben 1675 in Zittau)\newline}
	Der deutsche Organist und Komponist Andreas Hammerschmidt (auch „-schmied“ oder „-schmiedt“) floh mit seiner Familie vor der forcierten Rekatholisierung in Böhmen und gelangte so nach Freiberg im Kurfürstentum Sachsen, wo er vermutlich seine musikalische Ausbildung erhielt. Nach zwei vorausgehenden Organistenstellen ließ er sich schließlich 1639 als Organist in der damals reichen Stadt Zittau nieder, wirkte dort bis zu seinem Lebensende und gelangte zu gesellschaftlichem Ansehen und überdurchschnittlichem materiellem Wohlstand.
	
	Als Komponist des Frühen bis Mittleren Barock ist er in die Gruppe der evangelisch-lutherischen Kirchenkomponisten wie Heinrich Schütz und Johann Sebastian Bach einzuordnen. Sein kompositorisches Schaffen umfasst unter anderem Lieder, Kantaten, Motetten und Instrumental- wie auch Vokalkompositionen. Im Jahre 1757 vernichtete der große Stadtbrand in Zittau leider einen Großteil der Quellen über Hammerschmidt.
	
	Quelle: Deutsch- und englischsprachige Wikipedia

	
	\section*{Thomas Weelkes}
	\paragraph{(geboren 1576 in Elsted, West Sussex; gestorben 1623 in London)\newline}
	Der englische Komponist Thomas Weelkes wurde 1598 zum Organisten des Winchester College ernannt und wechselte 1601 in die Kathedrale von Chichester. Dort wurde er jedoch später entlassen, da er an der Orgel getrunken hatte und während des Gottesdienstes schmutzige Worte benutzte. Weelkes, ein Meister der Wortmalerei, war als Madrigalist bekannt, dessen zweiter Band (1600 veröffentlicht) eine der wichtigsten Sammlungen in der englischen Madrigal-Tradition ist.

	Quellen: emmanuelmusic.com,	Wikipedia, Duden
	
	
	\section*{Francis Poulenc}
	\paragraph{(geboren 1899, Paris; gestorben 1963, ebenda)\newline}
	Francis Jean Marcel Poulenc wuchs in einem musikalischen Haushalt auf und erhielt ersten Klavierunterricht durch seine Mutter. Nach weiterer Ausbildung als Pianist blieb das Klavier zunächst zentral und dominiert seine frühen Werke. Nicht nur Stravinsky, Chevalier und das französische Vaudeville beeinflussten ihn, sondern auch der Freund und Mentor Erik Satie, der sich dem damaligen französischen musikalischen Mainstream entzog, sowie seine Mitgliedschaft im Komponistenzirkel „Les Six“, der den Impressionismus zugunsten einer größeren Einfachheit und Klarheit ablehnte. So stand Poulenc beispielsweise Ravels musikalischen Ansichten kritisch gegenüber, wenngleich er ihn als Person sehr respektierte. Poulencs Bewusstsein für seine Insuffizienz in Bezug auf das technische Niveau seiner weitestgehend autodidaktisch erworbenen kompositorischen Fertigkeiten bewog ihn später zu Unterricht bei Charles Koechlin. Nach ab den 1920er Jahren – auch als Pianist – eintretendem internationalem Erfolg führten 1936 der Unfalltod eines Freundes und der Besuch des Wallfahrtsortes Rocamadour dazu, dass sich der Komponist intensiv auf den katholischen Glauben zurückbesann, mit der Komposition geistlicher Werke begann und eine stilistisch ernstere Komponente entwickelte. Die letzterer entspringenden Werke erhielten erst in den 1950er Jahren größere Anerkennung.
	
	Im Privatleben stellten der (während der Nazibesetzung auch politische) Umgang mit seiner Homosexualität sowie wiederkehrende, seine Schaffenskraft beeinflussende Depressionen Herausforderungen dar.
	
	Poulenc selbst verstand seinen Schaffensschwerpunkt in der Komposition von Opern. Kritiker sehen insbesondere die Dualität seiner stilistischen Pole: leichtherzig bis pietätlos versus ernsthaft-gewichtig. Das Gesamtwerk gilt als weitestgehend diatonisch und melodiezentriert und umfasst Bühnenwerke, Filmmusik, Geistliche Werke, weltliche Chorwerke, Kammermusik, zahlreiche Lieder sowie Klavier- und Orchesterwerke.
	
	Quelle: Deutsch- und englischsprachige Wikipedia
	
	\section*{Michael Praetorius}
	\paragraph{(geboren 1571 in Creuzburg bei Eisenach; gestorben 1621 in Wolfenbüttel)\newline}
	Michael Praetorius, eigentlich Michael Schulteis, war ein deutscher Komponist, Organist, Hofkapellmeister und Gelehrter im Übergang von der Renaissance- zur Barockzeit. Sein Vater war Pfarrer, der in Wittenberg bei Martin Luther studiert hatte. Auch die beiden älteren Brüder wurden Pfarrer, und so schrieb sich auch Michael Praetorius schon mit 11 Jahren an der Universität in Frankfurt/Oder für Theologie ein. Als der Vater und die Brüder starben, war niemand mehr da, der den 16-jährigen finanziell unterstützen konnte. Er wechselte auf die Orgelbank, um seinen Lebensunterhalt zu verdienen. Auch ohne geregelte musikalische Ausbildung erwarb er sich schnell einen überregionalen Ruf. So kam er an den Hof des musikliebenden Herzogs Heinrich Julius zu Braunschweig. 
	
	Am Hof in Wolfenbüttel und Gröningen stieg er zum Hofkomponisten auf und entfaltete in kurzer Zeit eine ungeheuer fruchtbare Nebentätigkeit als Komponist und Herausgeber. Er war ein glühender Anhänger der aufregenden musikalischen Neuerungen aus Italien am Übergang von der Renaissance zum Barock. Leider konnte er nie selbst nach Italien reisen, aber durch Kompositionen, Aufführungen und Veröffentlichungen wurde er zum Herold der Neuen Musik in Deutschland. Praetorius starb am 15. Februar 1621 und wurde unter der Orgelempore der Hauptkirche Beatae Mariae Virginis in Wolfenbüttel beigesetzt.
	
	\section*{Randall Thompson}
	\paragraph{(geboren 1899 in New York City; gestorben 1984 in Boston) \newline}
	Randall Thompson war ein US-amerikanischer Komponist und Musikpädagoge, der vor allem für seine Chorwerke bekannt wurde. Er studierte an der Harvard University und am American Conservatory in Rom. 1927 wurde er Professor für Musik, Organist und Chorleiter am Wellesley College. Später unterrichtete er an der Harvard University, leitete verschiedene Madrigalchöre und den New Yorker Juilliard Choir und wurde Professor an der Berkeley University, der University of Virginia und der Princeton University. Seine Studie College Music (1935) bewirkte eine grundlegende Reform der universitären Musikausbildung in den USA. Thompson komponierte drei Sinfonien, zwei Opern, weitere sinfonische Werke und Kammermusik. Populär wurden jedoch vor allem seine Chorwerke.
	
	
	\section*{Camerata Vocale München}
	Die Camerata Vocale München gründete sich 2016 auf Initiative von Clayton Bowman zunächst als Projektchor mit Sängerinnen und Sängern aus München und Heidelberg.
	
	Ihr erstes Projekt bestand in der Gestaltung eines Motettengottesdienstes am Palmsonntag in der ältesten evangelischen Kirche Münchens, der St. Paulus-Kirche im Stadtteil Perlach, mit Bachs Jesu, meine Freude. Im selben Jahr folgte ein Pfingstkonzert mit Chormusik aus fünf Jahrhunderten unter anderem von Reger, Purcell, Britten, Gesualdo, Whitacre, Vaughan Williams u.a., ebenfalls in der St. Paulus-Kirche Perlach. Im März 2017 erarbeitete die Camerata Vocale ihr bislang letztes Programm als Projektchor: Das Stabat Mater von Domenico Scarlatti, das Requiem von Ildebrando Pizzetti und das Crucifixus von Antonio Lotti wurden zur österlichen Fastenzeit in der Nikodemuskirche München und im Juli 2017 in der Lutherkirche in Mannheim im Rahmen einer Glockeneinweihung aufgeführt.
	
	Seit Sommer 2017 besteht die Camerata Vocale München als ambitionierter Kammerchor mit regelmäßiger Probenarbeit. Ihr gehören inzwischen 23 Sängerinnen und Sänger an. Nach kleineren Auftritten in Gottesdiensten der katholischen Hochschulgemeinde der LMU und in St. Paulus Perlach gab die Camerata im Januar 2018 ebendort ihr Debütkonzert mit A-capella-Werken u.a. von Reger, Brahms und Vaughan Williams. Im April 2018 wirkte sie an einer Produktion der Hochschule für Musik und Theater München mit, in deren Rahmen die Barockoper La Dafne von Marco da Gagliano im Rahmen der Barocktage 2018 im Großen Konzertsaal der HMT und in der Kreuzkirche Schwabing zur Aufführung gebracht wurde.
		
	\paragraph{Besetzung \newline}
	\begin{tabularx}\textwidth {lX}
	Sopran I & Denise Dudek \newline Cosima Stocker \\
	Sopran II & Kathrin Sollfrank \newline Ruth Stärcke \\
	Alt I & Nora Reinbold \newline Anna Hausner \\
	Alt II & Esther Stärcke \newline Nina Vinther \\
	Tenor I & Philipp Hummel \newline Julius Kiendl \\
	Tenor II & Benedikt Linder \newline (Chorvorstand) \newline Christoph Meinecke \\
	Bass I & Konrad Brückel \newline Tobias Gumpp \\
	Bass II & Joe Kübler \newline Andi Scharfstein
	\end{tabularx}

	\section*{Clayton Bowman}
	Clayton Bowman, geboren in Pittsburgh (USA), erhielt seinen ersten Klavierunterricht im Alter von 7 Jahren und wurde schon in jungen Jahren an die Chormusik herangeführt. Er sang in mehreren Auswahlchören und durfte bereits im Alter von 13 Jahren als Knaben-Solist sein Operndebüt in Benjamin Brittens "`Noye's Fludde"' mit dem Hartford Symphony Orchestra feiern.
	
	Erste Dirigiererfahrung sammelte Bowman während seiner Schulzeit. Bereits mit 14 Jahren leitete er als "`Student Conductor"' das symphonische Blasensemble an der Tolland Middle School und nahm im Jahr 2000 an einem Dirigiermeisterkurs an der University of South Carolina teil. Ab 2001 studierte er an der University of Connecticut Gesang und Musikwissenschaft mit Schwerpunkt Ensemblearbeit. Dieses Studium führte ihn schließlich nach Deutschland, wo er von 2004 bis 2008 Dirigieren bei Prof. Georg Grün, Klaus Thielitz und Wolfgang Seeliger an der Hochschule für Musik und Darstellende Kunst Mannheim studierte. Parallel dazu war Bowman lange Zeit Sänger und Dirigent zahlreicher Ensembles der Rhein-Neckar-Region.
	
	Als musikalischer Assistent beim Konzert Darmstadt und stellvertretender Dirigent des Universitätsorchesters Mannheim sammelte der junge Dirigent Erfahrung mit der Einstudierung großer Werke, wie beispielsweise Johann Sebastian Bachs "`Matthäuspassion"' und hatte zeitgleich die Leitung des Kammerchors Altrip, des Prot. Kirchenchors Mutterstadt und der Frauenchöre der Prot. Gemeinde in Dannstadt inne. Zusätzlich zu seinen Tätigkeiten als Dirigent, musizierte er häufig zusammen mit der Heidelberger Kantorei und der Hochschule für Kirchenmusik Heidelberg als Chorsänger und Solist. Darüber hinaus sang er regelmäßig im Kammerchor Saarbrücken, im Chor der Staatsphilharmonie Rheinland-Pfalz und gastierte gelegentlich als Chorist auf der Bühne des Nationaltheaters Mannheim und an der Oper Frankfurt.
	 
	Besonders aber mit dem Anglistenchor, einem der beiden Kammerchöre der Universität Heidelberg, hat sich Bowman einen Namen gemacht und eine große Leidenschaft für anspruchsvolle A-cappella-Musik entwickelt. Der Chor wirkte unter seiner Leitung vielfach im Ausland und arbeitete mit weltbekannten Chören wie dem Choir of Gonville und dem Caius College Cambridge zusammen.
	 
	Seit 2016 in München, leitet Bowman neben der vom ihm gegründeten Camerata Vocale München auch den Großen Chor des Akademischen Gesangvereins (AGV) und den Chor TonArt Sauerlach-Holzkirchen.
	
	\section*{Texte}
	
	\paragraph{Quatre motets pour le temps de Noël (FP152, 1952)\newline}
	
	I. O Magnum Mysterium\newline
	O magnum mysterium,\newline
	et admirabile sacramentum,\newline
	ut animalia viderent Dominum natum,\newline
	iacentem in praesepio.\newline
	Beata virgo cuius viscera meruerunt\newline
	portare Dominum Christum.\newline
	
	II. Quem Vidistis Pastores Dicite\newline
	Quem vidistis, pastores? Dicite,\newline
	annuntiate nobis, in terris quis apparuit?\newline
	Natum vidimus,\newline
	et choros Angelorum collaudantes Dominum.\newline
	Dicite quidnam vidistis?\newline
	et annuntiate Christi Nativitatem.\newline
	
	III. Videntes Stellam\newline
	Videntes stellam Magi,\newline
	gavisi sunt gaudio magno:\newline
	et intrantes domum,\newline
	obtulerunt Domino aurum, thus et myrrham.\newline
	
	IV. Hodie Christus Natus Est\newline
	Hodie Christus natus est:\newline
	hodie Salvator apparuit:\newline
	hodie in terra canunt Angeli,\newline
	laetantur Archangeli:\newline
	hodie exsultant iusti, dicentes:\newline
	Gloria in excelsis Deo, alleluia.\newline
	
		\paragraph{Hosanna to the Son of David\newline}
	Hosanna to the Son of David,\newline
	blessed be the King that cometh\newline
	in the name of the Lord, Hosanna.\newline
	Thou that sittest in the highest heav’ns,\newline
	Hosanna in excelsis Deo.
	
	\paragraph{Machet die Tore weit\newline}
	Machet die Tore weit\newline
	und die Türen in der Welt hoch,\newline
	daß der König der Ehren einziehe.\newline
	Wer ist der selbige König der Ehren?\newline
	Es ist der Herr,\newline
	stark und mächtig im Streit.\newline
	Machet die Tore weit\newline
	und die Türen in der Welt hoch!\newline
	Hosianna in der Höhe.\newline
	Hosianna dem Sohne Davids.
	
	\paragraph{Alleluia\newline}
	Alleluia.
		
	\end{multicols}
\end{document}