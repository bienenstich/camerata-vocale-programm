\documentclass[11pt, ngermanm, titlepage]{article}
\usepackage[ngerman]{babel}
\usepackage{fancyhdr}
\usepackage{multicol}
\usepackage{a4wide}
\usepackage{tabu, tabularx}

\setlength{\columnsep}{1cm}
\setlength{\columnwidth}{10cm}
\setlength{\extrarowheight}{15pt}

\begin{document}
	\pagenumbering{gobble}
	\begin{titlepage}
		\pagestyle{empty}
		\title{In dulci jubilo}
		\date{08.12.2018}
		\author{Camerata Vocale München, Ltg.: Clayton Bowman}
		\maketitle	
	\end{titlepage}
	\pagebreak
	\quad
	\pagebreak
	\section*{Programm}
	\begin{tabularx} \textwidth {lX}
		Francis Poulenc (1899 - 1963) & Quatre motets pour le temps de No\"el: \newline "`O Magnum Mysterium"'
		\\
		Thomas Weelkes (1576 - 1623) & "`Hosanna to the Son of David"' 
		\\
		Francis Poulenc (1899 - 1963) & Quatre motets pour le temps de No\"el: \newline "`Videntes Stellam"' 
		\\
		Andreas Hammerschmidt (1611 - 1675) & "`Machet die Tore weit"' 
		\\
		Francis Poulenc (1899 - 1963) & Quatre motets pour le temps de No\"el: \newline "`Quem Vidistis Pastores Dicite"' 
		\\
		Michael Praetorius (1571 - 1621) & "`In dulci jubilo"'
		\\
		Francis Poulenc (1899 - 1963) & Quatre motets pour le temps de No\"el: \newline "`Hodie Christus Natus Est"'
		\\
		Randall Thompson (1899 - 1984) & "`Alleluia"'
	\end{tabularx}
	\pagebreak
	\begin{multicols}{2}
	\section*{Komponist}
	\paragraph{Titel (1738)\newline}
	Nach einem dreijährigen Aufenthalt in New York kehrte Benjamin Britten im Jahre 1942 in seine Heimat England zurück. Während der Heimreise auf einem schwedischen Frachtschiff komponierte er neben anderen Werken seine "`Hymn to St. Cecilia"'. Zu Ehren der Patronin der Kirchenmusik, der Heiligen Cäcilia von Rom, waren bereits seit dem 17. und 18. Jahrhundert zahlreiche Werke entstanden. Britten führte diese Tradition fort und vertonte einen Text des Schriftstellers W. H. Auden, der sein Wirken zeit seines Lebens nicht nur auf persönlicher, sondern auch auf künstlerischer Ebene stark beeinflusste. 

	Ganz in der Tradition vorhergegangener Cäcilien-Loblieder schließt der erste anmutig schwebende Teil der dreigeteilten Hymne mit einer feierlichen Anrufung der Seligen: "`Blessed Cecilia, appear in visions to all musicians."' Diese Verse kehren nach jeder Sequenz musikalisch variiert, doch refrainartig wieder. Im zweiten Abschnitt, einem leichten, schnelleren Scherzo, spielen sich Soprane und Tenöre die Worte zu, während Alti und Bässe denselben Text als cantus firmus darbieten. Der Höhepunkt erfolgt im instrumental charakterisierten Schlussteil, in dem Soli aus jeder Stimme den Klang einzelner Instrumente beschreiben. Die letzte Invokation der Heiligen schließt mit einer erhabenen und friedvollen Kadenz in E-Dur. 
	
	\section*{Komponist}
	\paragraph{Titel (1738)\newline}
	Nach einem dreijährigen Aufenthalt in New York kehrte Benjamin Britten im Jahre 1942 in seine Heimat England zurück. Während der Heimreise auf einem schwedischen Frachtschiff komponierte er neben anderen Werken seine "`Hymn to St. Cecilia"'. Zu Ehren der Patronin der Kirchenmusik, der Heiligen Cäcilia von Rom, waren bereits seit dem 17. und 18. Jahrhundert zahlreiche Werke entstanden. Britten führte diese Tradition fort und vertonte einen Text des Schriftstellers W. H. Auden, der sein Wirken zeit seines Lebens nicht nur auf persönlicher, sondern auch auf künstlerischer Ebene stark beeinflusste. 
	
	Ganz in der Tradition vorhergegangener Cäcilien-Loblieder schließt der erste anmutig schwebende Teil der dreigeteilten Hymne mit einer feierlichen Anrufung der Seligen: "`Blessed Cecilia, appear in visions to all musicians."' Diese Verse kehren nach jeder Sequenz musikalisch variiert, doch refrainartig wieder. Im zweiten Abschnitt, einem leichten, schnelleren Scherzo, spielen sich Soprane und Tenöre die Worte zu, während Alti und Bässe denselben Text als cantus firmus darbieten. Der Höhepunkt erfolgt im instrumental charakterisierten Schlussteil, in dem Soli aus jeder Stimme den Klang einzelner Instrumente beschreiben. Die letzte Invokation der Heiligen schließt mit einer erhabenen und friedvollen Kadenz in E-Dur. 
	
	\section*{Komponist}
	\paragraph{Titel (1738)\newline}
	Nach einem dreijährigen Aufenthalt in New York kehrte Benjamin Britten im Jahre 1942 in seine Heimat England zurück. Während der Heimreise auf einem schwedischen Frachtschiff komponierte er neben anderen Werken seine "`Hymn to St. Cecilia"'. Zu Ehren der Patronin der Kirchenmusik, der Heiligen Cäcilia von Rom, waren bereits seit dem 17. und 18. Jahrhundert zahlreiche Werke entstanden. Britten führte diese Tradition fort und vertonte einen Text des Schriftstellers W. H. Auden, der sein Wirken zeit seines Lebens nicht nur auf persönlicher, sondern auch auf künstlerischer Ebene stark beeinflusste. 
	
	Ganz in der Tradition vorhergegangener Cäcilien-Loblieder schließt der erste anmutig schwebende Teil der dreigeteilten Hymne mit einer feierlichen Anrufung der Seligen: "`Blessed Cecilia, appear in visions to all musicians."' Diese Verse kehren nach jeder Sequenz musikalisch variiert, doch refrainartig wieder. Im zweiten Abschnitt, einem leichten, schnelleren Scherzo, spielen sich Soprane und Tenöre die Worte zu, während Alti und Bässe denselben Text als cantus firmus darbieten. Der Höhepunkt erfolgt im instrumental charakterisierten Schlussteil, in dem Soli aus jeder Stimme den Klang einzelner Instrumente beschreiben. Die letzte Invokation der Heiligen schließt mit einer erhabenen und friedvollen Kadenz in E-Dur.
	
	\section*{Camerata Vocale München}
	Die Camerata Vocale München gründete sich 2016 auf Initiative von Clayton Bowman zunächst als Projektchor mit Sängerinnen und Sängern aus München und Heidelberg.
	
	Ihr erstes Projekt bestand in der Gestaltung eines Motettengottesdienstes am Palmsonntag in der ältesten evangelischen Kirche Münchens, der St. Paulus-Kirche im Stadtteil Perlach, mit Bachs Jesu, meine Freude. Im selben Jahr folgte ein Pfingstkonzert mit Chormusik aus fünf Jahrhunderten unter anderem von Reger, Purcell, Britten, Gesualdo, Whitacre, Vaughan Williams u.a., ebenfalls in der St. Paulus-Kirche Perlach. Im März 2017 erarbeitete die Camerata Vocale ihr bislang letztes Programm als Projektchor: Das Stabat Mater von Domenico Scarlatti, das Requiem von Ildebrando Pizzetti und das Crucifixus von Antonio Lotti wurden zur österlichen Fastenzeit in der Nikodemuskirche München und im Juli 2017 in der Lutherkirche in Mannheim im Rahmen einer Glockeneinweihung aufgeführt.
	
	Seit Sommer 2017 besteht die Camerata Vocale München als ambitionierter Kammerchor mit regelmäßiger Probenarbeit. Ihr gehören inzwischen 23 Sängerinnen und Sänger an. Nach kleineren Auftritten in Gottesdiensten der katholischen Hochschulgemeinde der LMU und in St. Paulus Perlach gab die Camerata im Januar 2018 ebendort ihr Debütkonzert mit A-capella-Werken u.a. von Reger, Brahms und Vaughan Williams. Im April 2018 wirkte sie an einer Produktion der Hochschule für Musik und Theater München mit, in deren Rahmen die Barockoper La Dafne von Marco da Gagliano im Rahmen der Barocktage 2018 im Großen Konzertsaal der HMT und in der Kreuzkirche Schwabing zur Aufführung gebracht wurde.
	
	\paragraph{Besetzung \newline}
	\begin{tabularx}\textwidth {lX}
	Sopran I & Cosima Stocker \newline Denise Dudek \\
	Sopran II & Kathrin Sollfrank \newline Ruth Stärcke \\
	Alt I & Name Name \\
	Alt II & Name Name \\
	Tenor I & Philipp \\
	Tenor II & Name Name \\
	Bass I & Name Name \\
	Bass II & Name Name
	\end{tabularx}

	\section*{Clayton Bowman}
	Clayton Bowman, geboren in Pittsburgh (USA), erhielt seinen ersten Klavierunterricht im Alter von 7 Jahren und wurde schon in jungen Jahren an die Chormusik herangeführt. Er sang in mehreren Auswahlchören und durfte bereits im Alter von 13 Jahren als Knaben-Solist sein Operndebüt in Benjamin Brittens "`Noye's Fludde"' mit dem Hartford Symphony Orchestra feiern.
	
	Erste Dirigiererfahrung sammelte Bowman während seiner Schulzeit. Bereits mit 14 Jahren leitete er als "`Student Conductor"' das symphonische Blasensemble an der Tolland Middle School und nahm im Jahr 2000 an einem Dirigiermeisterkurs an der University of South Carolina teil. Ab 2001 studierte er an der University of Connecticut Gesang und Musikwissenschaft mit Schwerpunkt Ensemblearbeit. Dieses Studium führte ihn schließlich nach Deutschland, wo er von 2004 bis 2008 Dirigieren bei Prof. Georg Grün, Klaus Thielitz und Wolfgang Seeliger an der Hochschule für Musik und Darstellende Kunst Mannheim studierte. Parallel dazu war Bowman lange Zeit Sänger und Dirigent zahlreicher Ensembles der Rhein-Neckar-Region.
	
	Als musikalischer Assistent beim Konzert Darmstadt und stellvertretender Dirigent des Universitätsorchesters Mannheim sammelte der junge Dirigent Erfahrung mit der Einstudierung großer Werke, wie beispielsweise Johann Sebastian Bachs "`Matthäuspassion"' und hatte zeitgleich die Leitung des Kammerchors Altrip, des Prot. Kirchenchors Mutterstadt und der Frauenchöre der Prot. Gemeinde in Dannstadt inne. Zusätzlich zu seinen Tätigkeiten als Dirigent, musizierte er häufig zusammen mit der Heidelberger Kantorei und der Hochschule für Kirchenmusik Heidelberg als Chorsänger und Solist. Darüber hinaus sang er regelmäßig im Kammerchor Saarbrücken, im Chor der Staatsphilharmonie Rheinland-Pfalz und gastierte gelegentlich als Chorist auf der Bühne des Nationaltheaters Mannheim und an der Oper Frankfurt.
	 
	Besonders aber mit dem Anglistenchor, einem der beiden Kammerchöre der Universität Heidelberg, hat sich Bowman einen Namen gemacht und eine große Leidenschaft für anspruchsvolle A-cappella-Musik entwickelt. Der Chor wirkte unter seiner Leitung vielfach im Ausland und arbeitete mit weltbekannten Chören wie dem Choir of Gonville und dem Caius College Cambridge zusammen.
	 
	Seit 2016 in München, leitet Bowman neben der vom ihm gegründeten Camerata Vocale München auch den Großen Chor des Akademischen Gesangvereins (AGV) und den Chor TonArt Sauerlach-Holzkirchen.
	\section*{Texte}
	\end{multicols}
\end{document}